\aufgabe{Permutation Feature Importance}{
	
Permutation Feature Importance is one of the oldest and most widely used IML techniques. It was first suggested by Breiman in the random forest paper \cite{}.
It is defined as $PFI_j = \mathbb{E}[\mathcal{L}(Y, f(X_{-j}, \tilde{X}_j^{\emptyset}))] - \mathbb{E}[\mathcal{L}(Y, f(X))]$, where $\tilde{X}_j$ is an independent random variable that preserves the marginal distribution $P(X_j)$. We can approximate sampling from the marginal distribution by random permutation of the original feature's observations.

\begin{enumerate}
    \item Implement Permutation Feature Importance.
    \item Download the ... dataset. Fit a model of your choice. Apply PFI to the dataset and interpret the result. What insight into model and data do we gain?
    \item PFI has been criticized for its extrapolation. Under which assumptions does PFI lead to extrapolation, and when doesn't it?
    \item Prove that  $P(X) \neq P(\tilde{X}_j, X_{-j})$ if $X_j \not \perp X_{-j}$.
    \item Demonstrate the extrapolation problem empirically. Therefore, find a real-world dataset with dependent features and compare the distribution of the covariates $X$ before and after permutation.
    
\end{enumerate}

}
