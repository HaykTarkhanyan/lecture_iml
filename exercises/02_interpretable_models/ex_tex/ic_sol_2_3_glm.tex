\loesung{
%
%\begin{table}[ht]
%	\centering
%	\begin{tabular}{lcc|c}
%		\hline
%		 & Hepatitis A & no Hepatitis A & $\Sigma$\\
%		\hline
%		Salsa eaten & $218 (a)$ & $45 (b)$ & $263$ \\
%		Salsa not eaten & $21 (c)$ & $85 (d)$ & $106$\\
%		\hline
%		$\Sigma$ & $239$ & $130$ & $369$\\
%		\hline
%	\end{tabular}
%\end{table}

\begin{table}[ht]
	\centering
	\begin{tabular}{rrrrrr}
		\hline
		& WINTER & SPRING & SUMMER & FALL & $\Sigma$ \\ 
		\hline
		$y$=0 & 174.00 & 111.00 & 98.00 & 128.00 & 511.00 \\ 
		$y$=1 & 7.00 & 73.00 & 90.00 & 50.00 & 220.00 \\ 
		$\Sigma$ & 181.00 & 184.00 & 188.00 & 178.00 & 731.00 \\ 
		\hline
	\end{tabular}
\end{table}

\begin{enumerate}[a)]
	\item Odds for ``high number of bike rentals'' vs. ``low to medium number of bike rentals'' in winter: $$\text{odds} = \frac{P(y=1\, | \,\texttt{season}=\text{WINTER})}{P(y=0\, | \,\texttt{season}=\text{WINTER})} = \frac{7}{174} = 0.04 $$
	\textbf{Interpretation:} In winter the occurrence of \texttt{cnt} $> 5531$ ($y$=1) is $0.04$ times as likely as \texttt{cnt} $\leq 5531$ ($y$=0).
	\item Odds Ratio: 
	\begin{align*}
		\text{odds ratio} 
		&= \frac{P(y=1\, | \,\texttt{season}=\text{SPRING})\,/\,P(y=0\, | \,\texttt{season}=\text{SPRING})}{P(y=1\, | \,\texttt{season}=\text{WINTER})\,/\,P(y=0\, | \,\texttt{season}=\text{WINTER})} \\
		&= \frac{73/111}{7/174} = 16.35
	\end{align*}
	\textbf{Interpretation:} there is a $16.35$ times higher chance of having "high bike rentals" in season SPRING compared to the reference category (WINTER).
	\item Table:
	
	\begin{table}[ht]
		\centering
		\begin{tabular}{rrrr}
			\hline
			& Estimate & Std. Error & Pr($>$$|$z$|$) \\ 
			\hline
			(Intercept) & -3.2131 & 0.3854 & 0.0000 \\ 
			seasonSPRING & 2.7941 & 0.4138 & 0.0000 \\ 
			seasonSUMMER & 3.1280 & 0.4121 & 0.0000 \\ 
			seasonFALL & 2.2731 & 0.4199 & 0.0000 \\ 
			\hline
		\end{tabular}
	\end{table}

	The intercept gives the odds for ``high number of bike rentals'' vs. ``low to medium number of bike rentals'' in winter: exp$(-3.2131) = 0.04$. Interpretation as in a).
	
	Regarding the estimate of seasonSPRING: $\text{odds ratio (when season changes from winter to spring)} = \text{exp}(2.7941) =  16.35$. Interpretation as in b).
	\item Table:
	
	\begin{table}[ht]
		\centering
		\begin{tabular}{rrrr}
			\hline
			& Estimate & Std. Error & Pr($>$$|$z$|$) \\ 
			\hline
			(Intercept) & -8.5176 & 1.2066 & 0.0000 \\ 
			seasonSPRING & 1.7427 & 0.5977 & 0.0035 \\ 
			seasonSUMMER & -0.8566 & 0.7660 & 0.2635 \\ 
			seasonFALL & -0.6417 & 0.5543 & 0.2470 \\ 
			temp & 0.2902 & 0.0391 & 0.0000 \\ 
			hum & -0.0627 & 0.0124 & 0.0000 \\ 
			windspeed & -0.0925 & 0.0305 & 0.0024 \\ 
			days\_since\_2011 & 0.0166 & 0.0014 & 0.0000 \\ 
			\hline
		\end{tabular}
	\end{table}
	
	If all features are considered in the model, the $\beta$-value for the intercept is higher in absolute terms, but the odds changes to exp($-8.5176) = 0.0002$, i.e. the probability of ``high number of bike rentals'' is even less in winter when considering the full model compared to the one only containing feature \texttt{season}. Also the higher chance of having "high bike rentals" in season SPRING compared to WINTER declined to exp($1.7427)=5.71$ (vs. $16.35$ in the smaller model).
\end{enumerate}
}
