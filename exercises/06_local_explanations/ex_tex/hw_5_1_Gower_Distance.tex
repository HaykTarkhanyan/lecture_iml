\aufgabe{Gower Distance}{

Gower's distance is one of the most popular ways of measuring the similarity or
dissimilarity between observations in the presence of mixed-type variables

$ d_G(\xv^{(1)}, \xv^{(2)}) =  \frac{1}{p}\sum_{j = 1}^{p} \delta_G(x_j^{(1)}, x_j^{(2)}) $
$\textnormal{ with } \delta_G(x_j^{(1)}, x_j^{(2)}) =
\begin{cases}
\frac{1}{\widehat{R}_j}|x_j^{(1)}- x_j^{(2)}| & \text{if $x_j$ is numerical} \\
\mathbb{I}_{x_j^{(1)} \neq x_j^{(2)}} & \text{if $x_j$ is categorical}
\end{cases}.$

Despite its popularity, the Gower's distance has a major drawback. In the following
exercises, you should identify this problem and construct a solution.

\begin{enumerate}[a)]
\item Let us consider the following example that should illustrate the problem:
We have a dataset with two features - sex and age.
Sex has the values M = male and F = female, while age could have values between 15 and 90,
but in this example we only consider 3 different values (15, 58, 90).
Overall, the following feature combinations are possible:
(F, 15), (F, 58), (F, 90), (M, 15), (M, 58), (M, 90).
Imagine that we have an observation $\xv^{(1)}$ =(Sex$^{(1)}$, Age$^{(1)}$) = (F, 15).
For this observation we want to measure the distance to all other possible observations $\xv^{(2)}$.
Derive the Gower's distance $d_G$ and feature-wise distance $\delta_G$ per row.

\begin{table}[ht]
\centering
\begin{tabular}{ccc|ccc|c}
  \hline
 Sex$^{(1)}$ & Sex$^{(2)}$ & $\delta_G(\textnormal{Sex}^{(1)}, \textnormal{Sex}^{(2)})$ & Age$^{(1)}$ & Age$^{(2)}$ & $\delta_G(\textnormal{Age}^{(1)}, \textnormal{Age}^{(2)})$ & $d_G(\xv^{(1)}, \xv^{(2)})$ \\
  \hline
 F & F & & 15 & 15 & &\\
 F & F & & 15 & 58 & &\\
 F & F & & 15 & 90 & &\\
 F & M & & 15 & 15 & &\\
 F & M & & 15 & 58 & &\\
 F & M & & 15 & 90 & &\\
   \hline
\end{tabular}
\label{tab:probgow}
\end{table}

\item
Imagine that the observation $\xv^{(1)}$ = (Sex$^{(1)}$, Age$^{(1)}$) = (F, 15) is the data point we want to
find counterfactuals for and imagine that all the other observations (Sex$^{(2)}$, Age$^{(2)}$)
are the counterfactual candidates.
Given the fully filled out table, which counterfactuals would you prefer?
Do you see a tendency towards a group of counterfactuals with respect to the
type of features? Describe why this is a problem.

\item An enhancement of the Gower's distance is to use different weights $w_j$ for the
different features $j \in \{1, ..., p\}$:
$ d_G(\xv^{(1)}, \xv^{(2)}) =  \frac{1}{p}\sum_{j = 1}^{p} w_j \delta_G(x_j^{(1)}, x_j^{(2)})$.
How would you choose $w_j$ in order to solve the above problem?
Think about possible weighting schemes.

\end{enumerate}
}
