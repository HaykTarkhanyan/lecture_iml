\loesung{
\begin{enumerate}[a)]
\item 
Overall, all customers, regardless of their personal status and gender, have on 
average a high probability of being a low (good) risk for the bank. 
The average marginal prediction for divorced or separated male customers
reveals a slightly higher risk for this group. 
\item 
The ALE is faster to compute and unbiased. Unbiasedness means that it does 
not suffer from the extrapolation problem which is especially apparent in PDPs 
when features are correlated.
% SAY in CLASS:
% Since \texttt{personal\_status\_sex} is a categorical feature, we cannot 
% use the Pearson correlation coefficients (even for 
% numerical features it is not always a good idea (see in-class exercise)). 
% For two categorical features we can use a $\Chi^2$-test 
% and for categorical vs. a numerical features we could rely on a one-way ANOVA F-test. 
\item 
The following pseudocode computes the pairwise sum of the absolute differences of relative 
frequencies in the categories of a categorical feature $x_j$ based on a 
feature $x_k$

%\lstinputlisting{"rsrc/get_diff_cat().R"}

\begin{algorithm}[h]
\caption{\texttt{get\_diff\_cat()}}
\begin{algorithmic}[1]
\Require \texttt{feature.k}: values of categorical feature for which relative frequencies per class are calculated
\Require \texttt{feature.j}: values of categorical feature for which similarity based on \texttt{feature.k} should be assessed
\State \texttt{dists} $\gets$ unique class combinations of \texttt{feature.j}
\State \texttt{x.count} $\gets$ number of observations per class of \texttt{feature.j}
\State \texttt{A} $\gets$ relative cross table of \texttt{feature.j} and \texttt{feature.k} weighted by \texttt{x.count} (per class of \texttt{feature.j} relative frequencies of classes of \texttt{feature.k} should sum up to 1)
\State \texttt{dist} $\gets$ sum up distances of probability distributions per unique class combination specified in \texttt{dists}
\State \textbf{return} \texttt{dist}, \texttt{dists}
\end{algorithmic}
\end{algorithm}

For our task at hand, we obtain the following distances
% \lstinputlisting{"rsrc/run_get_diff_cat.R"}
\lstinputlisting{"rsrc/run_get_diff_cat.txt"}

Overall, the following ordering of \texttt{personal\_status\_sex} was returned by the method: 
% \lstinputlisting{"rsrc/run_order_levels.R"}
\lstinputlisting{"rsrc/run_order_levels.txt"}


The ordering seems to be feasible, since categories including females are close to each other 
and also categories with males. 
Also the ordering of males according to their relationship status seems to make sense, since typically 
the process is: single, then married and then divorced :-).

\item \textbf{Bonus:} ALE and PDP are global interpretation tools which base their insights on averages (of predictions or 
prediction differences) over whole test sets.
Indeed vulnerable groups are typically not the majority of a population but have a low proportion, 
and biases might be overlooked.
Therefore, local explanation tools should be consulted, in addition to these methods in order to 
identify pointwise biases or discriminatory behavior.

\end{enumerate}
}