\loesung{

\begin{enumerate}
  \item 
  Both PDP and ALE plots show a strong linear effect of $x_1$, 
  where higher values of $x_1$ lead to higher values of predicted value.
  The PDP and ALE plot of $x_2$ show a strong decreasing effect of $x_2$ on the prediction. 
  The PDP of $x_2$ shows a steep jump for large values of $x_2$, while 
  the ALE plot shows a strong linear effect over the whole value range of $x_2$.
  Interpretation at $x_1 = 0.5$: 
  \begin{itemize}
    \item PDP: the model predicts on average a value of around $2.3$ for $y$ if for all data instances $x_1 = 0.5$.
    \item ALE: the model predicts on average an increase of around $2.5$ of $y$ for data instances with $x_1 = 0.5$ 
    compared to the average prediction.
    \end{itemize}
  \item 
  PDPs assume that features are uncorrelated. We know from the
  GAM output above - as well as the scatter plot - that $x_1$ and $x_2$ are
  highly correlated. Since PDPs extrapolate over predictions of artificial points that 
  are out of distribution, the interpretations might be misleading - especially in areas
  with low data density (high values of $x_2$) and if the model contains interactions.
  ALE on the other hand, does not predict in regions that are far away from the 
  training data and therefore do not suffer from the extrapolation issue of 
  PDPs. 
  \end{enumerate}
  
}
