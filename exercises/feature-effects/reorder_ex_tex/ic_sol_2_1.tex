\loesung{
\begin{enumerate}
  \item Which of the following statement(s) apply to feature effect methods? 
        \begin{enumerate}
        \item The value of the PDP at a point $x_j$, corresponds to the point-wise average of the values of the ICE curves at this point. $\Rightarrow$ \textbf{Correct}
        \item The PDP of a feature provides information about possible interaction effects of the feature. $\Rightarrow$ \textbf{Wrong}
        \item ICE curves of a feature for multiple data points provide information about possible interaction effects of the feature with others. $\Rightarrow$ \textbf{Correct}
        \item If we center the ICE/PDPs for categorical features, the expected changes always refer to a selected reference category. $\Rightarrow$ \textbf{Correct}
        \item ALE plots are based on conditional distributions, PDPs on marginal distributions. $\Rightarrow$ \textbf{Correct}
        \item If features are uncorrelated, ALE plots are equal to PDPs. $\Rightarrow$ \textbf{Wrong}
        \item ALE plots are faster to compute than PDPs if they are based on the same grid. $\Rightarrow$ \textbf{Correct}
        \end{enumerate}

\item You fitted a model that should predict the value of a property depending on 
    the number of rooms and square meters. 
    You want to compute feature effects using the following methods: 
    PDP, M-plots and (uncentered) ALE plots. 
    Which of the following strategies reflect which method? \\
    The feature effect for a 30 m$^2$ corresponds to... 
\begin{enumerate}[a)]
  \item ... what the model predicts on average for flats that also have around 30 m$^2$, for example, 28 m$^2$ to 32 m$^2$. $\Rightarrow$ \textbf{M-plot}
  \item ... how the model's predictions change on average when flats with 28 m$^2$ to 32 m$^2$ have 32 m$^2$ vs. 28 m$^2$. $\Rightarrow$ \textbf{uncentered ALE}
  \item ... what the model predicts on average if all properties in the dataset have 30 m$^2$. $\Rightarrow$ \textbf{PDP}
\end{enumerate}

\end{enumerate}
}