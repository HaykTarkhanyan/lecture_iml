\aufgabe{LIME}{
LIME is a local feature effect method that is based on locally approximating the decision function using a linear model. Then, the linear model coefficients are interpreted as usual. Like SHAP, LIME is an additive attribution technique.
\begin{enumerate}
    \item Apply a LIME implementation to the dataset from Exercise \ref{ex:shap}.\\ \textit{Hint: implementations are available in \href{https://github.com/marcotcr/lime}{python} and \href{https://cran.r-project.org/web/packages/lime/index.html}{R}.}
    \item Try different hyperparameter configurations (for instance, modify the kernel width). How does that influence the result?
    \item Compare the result to SHAP. In what way are LIME and SHAP similar? How do the methods differ?
\end{enumerate}
}
